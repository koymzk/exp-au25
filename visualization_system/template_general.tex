\documentclass[a4paper,12pt]{article}
% LuaLaTeX 日本語対応
\usepackage{luatexja}
\usepackage[ipaex]{luatexja-preset}
\usepackage{graphicx}
% LuaLaTeX-specific packages
\usepackage{fontspec}
\usepackage{luaotfload}
\usepackage{microtype}
\usepackage{tabularx}
\usepackage{titlesec}
\usepackage{geometry}
\usepackage{amsmath}
\usepackage{siunitx}
\usepackage{tikz}
\usetikzlibrary{external}
% 外部化をオンに(キャッシュ先を figures/ に指定)
\tikzexternalize[prefix=figures/]
\tikzexternalize[mode=only pdf]
\usepackage{pgfplots}
\pgfplotsset{compat=1.18}
\usepackage{float}
\usepackage{gnuplottex}
% --- redefine captions and Japanese labels ---
\usepackage[labelfont=bf]{caption}
\renewcommand{\figurename}{図}
\renewcommand{\tablename}{表}
% --------------------------------------------
\geometry{left=25mm,right=25mm,top=25mm,bottom=25mm}
\titleformat{\section}{\large\bfseries}{\thesection}{1em}{}
\titleformat{\subsection}{\normalsize\bfseries}{\thesubsection.}{1em}{}

\begin{document}

\title{映像化システム 実験レポート}
\author{氏名:山崎恒 \\班番号:2班E\\ 学籍番号:62221253}
\date{\today}

\maketitle

\newpage

\section{目的}
\begin{itemize}
    \item CTの動作原理を理解する.
    \item シミュレーションおよび実測データから断層像を再構成してみて,アルゴリズムの理解を深める.
\end{itemize}

\section{理論}

\section{実験方法}
\subsection{X 線CTのシミュレーション方法}
\subsection{光CTの実験装置}
\subsection{光CTのシミュレーション方法}
\subsection{A/D 変換の実験装置}

\section{実験結果と考察}
\subsection{X 線CT のシミュレーション}
\subsubsection{断層像とX 線による投影データ、サイノグラムの関係}
\subsubsection{画像再構成におけるフィルタ補正の効果}
\subsubsection{投影点数による再構成画像の変化}

\subsection{光CT のシミュレーション}
\subsubsection{光CT とX 線CT の比較}
\subsubsection{光CT 画像の性質}

\subsection{A/D 変換器を用いた正弦波信号の取得}

\end{document}